
% Set plain page style for ToC/LoF/LoT spanning more than one page.
\AfterTOCHead{\pagestyle{plain}}
\AfterStartingTOC{\clearpage}

% Style ToC/... Header:
% Store the name in old macro to be used in styled version.
\let\oldcontentsname\contentsname
\renewcommand{\contentsname}{%
  \huge\sf \rule{\textwidth}{.8pt}\\[\medskipamount]%
  \oldcontentsname\\[-2\medskipamount]
  \huge\sf \rule{\textwidth}{.8pt}
}
\let\oldlistfigurename\listfigurename
\renewcommand{\listfigurename}{%
  \huge\sf \rule{\textwidth}{.8pt}\\[\medskipamount]%
  \oldlistfigurename\\[-2\medskipamount]
  \huge\sf \rule{\textwidth}{.8pt}
}
\let\oldlisttablename\listtablename
\renewcommand{\listtablename}{%
  \huge\sf \rule{\textwidth}{.8pt}\\[\medskipamount]%
  \oldlisttablename\\[-2\medskipamount]
  \huge\sf \rule{\textwidth}{.8pt}
}

\newcommand{\STATEMENTPAGE}{
  ~ \vspace*{2cm} ~

  \textbf{Eidesstattliche Erkl\"arung}

  Ich erkl\"are hiermit an Eides Statt, dass ich die vorliegende Arbeit
  selbst\"andig verfasst und keine anderen als die angegebenen Quellen und
  Hilfsmittel verwendet habe.

  \textbf{Statement in Lieu of an Oath}

  I hereby confirm that I have written this thesis on my own and that I
  have not used any other media or materials than the ones referred to in
  this thesis.


  ~ \vspace{5cm} ~

  \textbf{Einverst\"andniserkl\"arung}

  Ich bin damit einverstanden, dass meine (bestandene) Arbeit in beiden
  Versionen in die Bibliothek der Informatik aufgenommen und damit
  ver\"offentlicht wird.

  \textbf{Declaration of Consent}

  I agree to make both versions of my thesis (with a passing grade)
  accessible to the public by having them added to the library of the
  Computer Science Department.

  \vspace*{5cm}
  \noindent\rule{8cm}{.4pt}\\[\medskipamount]
  \CITY, \SUBMISSIONDAY\ \SUBMISSIONMONTH, \SUBMISSIONYEAR

  \cleardoublepage
}

\ifthenelse{\boolean{logobottom}}{
  \newcommand*\logox{0}
  \newcommand*\logoy{-350}
  \newcommand*\logoopacity{.3}
  \newcommand*\authorsepwidth{2cm}
  \newcommand*\titlesepwidth{1cm}
  \newcommand*\pretitlewidth{0cm}
}{
  \newcommand*\logox{0}
  \newcommand*\logoy{0}
  \newcommand*\logoopacity{.1}
  \newcommand*\authorsepwidth{1cm}
  \newcommand*\titlesepwidth{0cm}
  \newcommand*\pretitlewidth{10cm}
}

% Punish widows and orphans because we're awful people:
\widowpenalty10000
\clubpenalty10000
% Allow for page breaks in long equations:
\allowdisplaybreaks
% Enable emergency stretches because we don't mind a few empty points.
\emergencystretch=10pt

\date{\CITY, \SUBMISSIONMONTH\ \SUBMISSIONYEAR}
\constructtitle{\vspace{\pretitlewidth}\Huge\TITLE\vspace{\titlesepwidth}}
\ifthenelse{\boolean{dissertation}}{
  \author{Dissertation zur Erlangung des Grades \\ 
  des Doktors der Naturwissenschaften \\
  der Fakultät für Mathematik und Informatik \\
  der Universität des Saarlandes \\[\authorsepwidth]
  vorgelegt von  \\
  \AUTHOR}
}{
  \author{%
    \Large %
    \UNIVERSITY\\[\medskipamount]
    \DEPARTMENT\\[.9\authorsepwidth]
    \textsc{\THESISKIND's Thesis} \\[.9\authorsepwidth]
    \emph{submitted by} \\
    \AUTHOR\\[.9\authorsepwidth]
  }
}

\setdescription{topsep=10pt,itemsep=0pt}

\addtoreflist{chapter}
\addtoreflist{section}

%%%%%%%%%%%%%%%%%%%%%%%%%%%%%%%%%%%%%%%%%%%%%%%%%%%%%
%%%%%%%%%%%%%%%%%%%%%%%%%%%%%%%%%%%%%%%%%%%%%%%%%%%%%
%%% Night Mode Configuration

\ifthenelse{\boolean{nightmode}}{
  \pagecolor[rgb]{0.3,0.3,0.3}
  \color[rgb]{1,1,1}
  \definecolor{bluekeywords}{RGB}{120, 170, 250}
  \definecolor{greentypes}{RGB}{167, 247, 101}
  \definecolor{redstrings}{RGB}{247, 211, 92}
  \definecolor{graynumbers}{rgb}{0.8, 0.8, 0.8}
  \definecolor{goldcomments}{RGB}{247, 211, 92}
  \renewcommand*\logoopacity{1}
}{
  \definecolor{bluekeywords}{rgb}{0.13, 0.13, 1}
  \definecolor{greentypes}{rgb}{0, 0.5, 0}
  \definecolor{redstrings}{RGB}{171, 114, 2}
  \definecolor{graynumbers}{rgb}{0.5, 0.5, 0.5}
  \definecolor{goldcomments}{rgb}{0.6, 0.4, 0.08}
}

%%%%%%%%%%%%%%%%%%%%%%%%%%%%%%%%%%%%%%%%%%%%%%%%%%%%%
%%%%%%%%%%%%%%%%%%%%%%%%%%%%%%%%%%%%%%%%%%%%%%%%%%%%%
%%% Headers and Footers

% Change \leftmark for chapters to be of form e.g. `1. Introduction`.
\renewcommand{\chaptermark}[1]{\markboth{\thechapter.~ #1}{}}
% Change \rightmark for sections to be of form e.g. `1.2. Related Work`.
\renewcommand{\sectionmark}[1]{\markright{\thesection. ~#1}}

% Set chapter numbers to Helvetica.
\ChNumVar{\fontsize{60}{62}\usefont{OT1}{phv}{m}{n}\selectfont}
% lmss in bold is as close as we can get to the section font.
\ChTitleVar{\fontsize{30}{33}\usefont{OT1}{lmss}{b}{n}\selectfont}

% Header is now in small caps and larger.
\addtokomafont{pagehead}{\large\scshape}

% On even pages, put the chapter title into the left header
\lehead{\leftmark}
% On odd pages, put the section title into the right header
\rohead{\rightmark}
% Clear footer right and left, put page number in center
\lefoot*{}
\rofoot*{}
\cfoot*{\pagemark}


\newcolumntype{L}{>{$}l<{$}} % math-mode version of "l" column type
\newcolumntype{R}{>{$}r<{$}} % math-mode version of "r" column type
\newcolumntype{C}{>{$}c<{$}} % math-mode version of "c" column type

\setlength{\marginparwidth}{2.8cm} % Set the margin par width appropriately
\renewcommand\arraystretch{1.5} % Increase distance between array rows
\setlength{\jot}{8pt}

%%%%%%%%%%%%%%%%%%%%%%%%%%%%%%%%%%%%%%%%%%%%%%%%%%%%%
%%%%%%%%%%%%%%%%%%%%%%%%%%%%%%%%%%%%%%%%%%%%%%%%%%%%%
%%% Define custom language, examplified by RTLola

\lstdefinelanguage{Lola}{
  keywords=[0]{input, output, trigger},
  keywordstyle=[0]\bfseries\color{bluekeywords},
  keywords=[1]{if, then, else, aggregate, defaults, offset},
  keywords=[2]{Int8, Int16, Int32, Int64, UInt8, UInt16, UInt32, UInt64, Bool, Float32, Float64, @1Hz, @5Hz, @10Hz, @100mHz, @1kHz},
  keywordstyle=[2]\color{greentypes},
    sensitive=false,
    comment=[l]{//},
    morecomment=[s]{/*}{*/},
    morestring=[b]',
    morestring=[b]"
}

\lstset{
    autogobble,
    language={Lola},
    columns=fullflexible,
    showspaces=false,
    showtabs=false,
    breaklines=true,
    showstringspaces=false,
    breakatwhitespace=true,
    escapeinside={(*@}{@*)},
    commentstyle=\color{greencomments},
    keywordstyle=[1]\color{bluekeywords},
    stringstyle=\color{redstrings},
    numberstyle=\color{graynumbers},
    basicstyle=\ttfamily\footnotesize,
    frame=l,
    framesep=12pt,
    xleftmargin=12pt,
    tabsize=4,
    captionpos=b,
    mathescape,
}

% Language without any highlighting.
\lstdefinelanguage{none}{
  identifierstyle=
}